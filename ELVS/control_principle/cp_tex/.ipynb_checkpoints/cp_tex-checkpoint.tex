\documentclass[reprint,unsortedaddress,amsmath,amssymb,aps,prl,showkeys]{revtex4-2}
\usepackage{graphicx}% Include figure files
\usepackage{dcolumn}% Align table columns on decimal point
\usepackage{subfigure}
\usepackage{bookmark}
% \usepackage{biblatex}
\usepackage{float}
\usepackage{url}
\usepackage{bm}% bold math
\usepackage{hyperref}% add hypertext capabilities
\usepackage[mathlines]{lineno}% Enable numbering of text and display math

\begin{document}

\title{Controlling principle for spatial epidemic}
\author{Gezhi Xiu}
% \author{Yu Liu}
% \email{liuyu@urban.pku.edu.cn}
\affiliation{Institute of Remote Sensing and Geographic Information Systems (IRSGIS), Peking University}
\date{\today}

\begin{abstract}
    The epidemic threshold has always been a core concern in epidemiology. We analytically prove that the control principles in spatial epidemiology.
\end{abstract}
\maketitle
\section{Introduction}

This is good.

\section{Discussion}

% Future endeavor

\end{document}