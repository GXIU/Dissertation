\chapter{Spreading patterns: An Allee effect Perspective}

The spatio-temporal evolution of information, dialect, prosperity is a key to understanding urban developments. However, the allometric growth of the mentioned topics remain unexplained. 

A possible explanation is the surface tension. In~\cite{PhysRevX.7.031008}, J. Burridge explained the spatial evolution of human dialects with a simple spatial model with somehow predictability. The model shows that the language dialect boundaries separated regions are controlled by a length minimizing effect analogous to surface tension. This work set up a fundamental backbone for language evolution.

Many factors, such as co-existence and unmatched reproductions can lead to the so-called Allee effect in natural population expansion. The Allee effect is characterized by a decrease in the per capita growth rate at low densities, which can be due to increased mortality coming from interspecific competition or reduced fitness due to suboptimal mating opportunities. Such effect is known to affect the rate of spatial spread of a population, and is expected to modify genetic drift on the edge of the studied population~\cite{alleeevidence,spatialwaveallee}.

In urban systems, the spreading process is the key mechanism to understand a city as a whole, and why urbanization is the basic theme of today's world. To study the spreading process, we begin with a version of two-dimensional reaction-diffusion equations~\footnote{A method for population dynamics since \cite{skellam1951random}.}:\begin{align}
    \frac{\partial u}{\partial t} = \frac{\partial^2 u}{\partial x^2} + f(u),\label{rdeq}
\end{align} for $t>0$ representing time, $x\in \mathbf{R}^2$ standing for position, and $u = u(t,x)$ is the density of population. Its evolution in time is controlled by the combined effects of the diffusion and local reproduction as a function of $f$. The solution of Eq.~\ref{rdeq} converge to travelling wave solutions, describing the invasion towards unoccupied region in constant speed and density profile $U$. It helps to answer how do the proportions of the different fractions evolve in a travelling wave generated by (i) a classical KPP model, or (ii) the presence of an Allee effect modification.

\section{General Model}

Assuming that the population of a city is composed of \textit{genes}, whose total density $u$ obbeys Eq.~\ref{rdeq}, and made of several neutral fractions $v^k$. At time $t=0$, \begin{align}
    u_0(x) = \sum_{k\geq 1} v_0^k(x).
\end{align}
We also assume that the genes in each fraction only differ by their allele and their position, and the density of each fraction follows that\begin{equation}
    \left\{\begin{array}{l}
    \frac{v^{k}}{\partial{t}}=\frac{\partial v^{k}}{\partial x^2}+v^{k} g(u), t>0, x \in \mathbf{R} \\
    v^{k}(0, x)=v_{0}^{k}(x), x \in \mathbf{R}
    \end{array}\right.
\end{equation} where $g(u)$ is defined as the per capita growth rate $f(u)/u$. 

\paragraph{The growth function $f$.} 
$f$ is assumed to be vanished at $0$ and $1$, with $f'(1)<0$ as two stationary states. A KPP type of growth function is defined to be of the form~\begin{align}
    0<f(u)\le f'(0) u, \qquad \forall u \in (0,1).
\end{align} This type of growth function is controlled by a logistic peer pressure; The second type of growth function is a cubical polynomial,\begin{align}
    f(u) = u(1-u)(u-\rho), \qquad \forall u \in (0,1).
\end{align} Thus, the growth rate $g(u) = f(u)/u < 0$ for small $u$'s, which corresponds to a strong Allee effect. $\rho$, usually referred as the Allee threshold, below which the growth rate $<0$.

By solving Eq.~\ref{rdeq} that satisfies $u(t,x) = U(x-ct)$ for a speed $c>0$, we receive the travelling wave solutions. To be specific, Eq.~\ref{rdeq} is transferred into the following differential equation:\begin{equation}
    \left\{ \begin{array}{l} U^{\prime\prime} + cU' + f(U) = 0,\\
        U(-\infty) = 1,\\
        U(\infty) = 0
    \end{array} \right.
\end{equation}
With the Allee effect, we have $U(x) = 1/(1+\exp\{x/\sqrt{2}\})$ and $c = (1-2\rho)/\sqrt{2}$.

\section{Pulled and Pushed waves}

As mentioned in Chapter~\ref{Chap:sy}, the spatial spreads in cities results from different sources. Here, we introduce a different formation of range expansions: Pulled and Pushed waves driven by the Allee effect.