\chapter{Social Segregations}

\section{Background and Intuitions}

Agent-based models are essential tools for social sciecnes since 1971, Schelling proposed his original work about social segregations~\cite{schelling1971dynamic}, which is still a fundamental standing point for many principles in urban ecology. In Schelling's model, families of two types inhabit blocks in a finite squre, leaving out a proportion empty. Each family's neighbour is considered as a $5\times 5$ square centered at his location. Through different rules, families move if their neighbors consist of too may of the opposite type. If people have a preference for inhabiting with those of their color, the movements of individual families led to complete segregation invariably. 

The mathematical proofs about Schelling's models are extremely hard and has only been carried out in 1D version, but still inspiring. The average size of monochromatic neighborhoods is proved to be independent of population size $n$ and polynomial in the size-definition of neighbourhood $w$~\cite{schelling1971dynamic}. Durrett studied the metapopulation version of Schelling model~\cite{durrett2014exact}, reaching out the critical threshold of multichromatic neighbours. For more information about probability theory and life, please reference~\href{https://www.stat.berkeley.edu/~aldous/157/}{Probability and the Real World}.


\section{Beyond Places: Behavioural Segregations}

During the COVID-19 pandemics, cities are the places that mostly suffers. The anti-epidemic measures varies across the world. 

Small residential areas, or even houses are considered as the basic unity of quarantine areas. 