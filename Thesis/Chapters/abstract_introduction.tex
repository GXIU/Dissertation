\chapter*{abstract}

The past thirty years have witnessed the influences that data and physics have brought to urban studies. A dynamical perspective of cities is constantly refreshing our eyeballs. However, despite of the information and dynamical models, we still have little understanding of where these disequilibrium dynamics of the urbanization process are leading us. In this thesis, we made some initial attempts to understand the cities as a mathematical ecosystem.

We first give a stochastic and out-of equilibrium model of multi-city growth that describes how the cities co-develop in a given area with limited shared resources. The model demonstrated the occurrence of 

The model explains the appearance of secondary subcenters as an effect of traffic congestion, and predicts a sublinear increase of the number of centers with population size. Within the framework of this model, we are further able to give a prediction for the scaling exponent of the total distance commuted daily, the total length of the road network, the total delay due to congestion, the quantity of CO2 emitted, and the surface area with the population size of cities. In the third part, we focus on the quantitative description of the patterns of residential segregation. We propose a unifying theoretical framework in which segregation can be empirically characterised. In the fourth and last part, we succinctly present the most important---theoretical and empirical---results of our studies on spatial networks.
Throughout this thesis, we try to convey the idea that the complexity of cities is -- almost paradoxically -- better comprehended through simple approaches. Looking for structure in data, trying to isolate the most important processes, building simple models and only keeping those which agree with data, constitute a universal method that is also relevant to the study of urban systems.

\chapter{Introduction}


\section{Agent-based cities}




\section{Urban observations}




\section{Methodology}





\section{About the Thesis}