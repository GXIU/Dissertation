\chapter{研究内容}

根据对流行病在人群中传播建立的微分方程,我们将研究如下几点内容:

\begin{itemize}
    \item 描述(尤其是有潜伏期的)传染病在人群中、网络中、广义地理空间的传播过程。
    \item 预报传染病高潮在复杂地理空间中到来的时刻。
    \item 对于存在疫情的城市建立安全区域与安全交通网络。
\end{itemize}

即SIS模型中,在给定人群数量是$N$,且人口充分混合的情况下,被感染人口只会稳定在一个固定的比例$I_e$左右;而一旦被感染者消失,$I=0$,人群中就永远不会再出现感染者了。这是隔离的数学依据:总人口为$N$的群体,在某一个瞬间有$I$个感染者,且我知道ta们是谁(如果不全知道,可以暂且将疑似患者当作病人);那么我将完全健康的人分为一组$S_1$,不确定是不是健康的人分为一组$I_1$,两组人不可以见面(将$I_1$组隔离)。过一段时间之后,$S_1$依然是完全健康的,$I_1$作为一个新的群体,健康的人会稳定在$\gamma/\beta$倍的$I_1$左右。同样的,我将这些完全健康的人分为$S_2$组,另外的不确定健康的人分为$I_2$组,进行隔离。我可以保证健康为$S_1+S_2$。依此类推,最终达到没有活体感染者的时候,疾病就消失了。

隔离代表着充分混合的反面,即交互为0。隔离可以不让传染病传染两侧的人。隔离的在数学上的本质意义在于划分出一个确定的,总数为定值(记为$N$)的人群。无论什么样的**充分混合**人群,初始状态的得病人数是多少,稳定状态都是一个只跟该疾病的**传染速率**和**恢复速率**有关系的固定比例。下文里面我们记这个比例为$(1-\gamma/\beta) = p\%$。隔离掉得病比例高于该比例$p\%$的人群有助于疾病的防控;而其他的情况,固定人群数量的时候,患病人数几乎都会达到稳定状态$p\%$。所以大规模隔离往往是无效的,因为我们能见到的流行病,发病率往往远远小于$p\%$,这般隔离,如果没有救治,会使得患病人口最终达到$p\%$。好的解决办法只有是分离出高于该比例的小区域,这样才是有意义的隔离。这也是细致地图的重要用途。需要GISers的努力。连通的无疾病区域,无论多大都是安全的;分离高比例患病人群的区域,有助于患病比例自然降低,也方便集中救治。这应该是我们政策的导向:使得更大确定没有病的人,在更广阔的空间上自在的生产生活。

异质性人口建模目前主要集中在复杂网络上的流行病传播方面。研究目标大多是稳定人口与网络结构的关系。这种问题的探讨沟通了“完全没有沟通的隔离状态”和“充分混合的普通状态”,是一种中间状态。这种建模方式通过网络连边来限制可能接触到的人。

网络上疾病的传播也是一个经典的建模例子:假设每个地物是网络的一个结点,每个地物附近有着一些人口。结点之间存在着交互,也伴随着疾病的传播。无标度网络极易传播流行病,甚至没有临界人口。人群按照无标度连接很有可能全员感染。我们可以由此推断,城市建设走最高效率的无标度建设方式是行不通的。空间上这个问题还非常有待解决。重新建模是必须的。对于流行病学来说,临界人口的确定是很重要的事情。也是我们了解一个疾病需要做的第一件事。本质上来说,这个阈值是一个“密度”,或者是“强度”。只有确认了充分混合的人口总数,我们才可以确定这个数字$p$。所以可能可以做的事情,是直接用局部异质性人口密度$\rho_S$、$\rho_I$、$\rho_R$来取代原有的纯量关系。进而用移动性模式来确认传播规律,更进一步,可以通过限制/指示各个人群的交互强度来分化高患病密度人口。



\section{研究框架}



\section{研究内容}

\subsection{流动配置问题}

静态资源配置问题已经被广泛研究。我们在这里面向问题的另一个方向,即有目标的流配置问题。该问题受到基础设施的空间分布、人口固有密度、移动性加权等问题的影响,体现出极度复杂的特性。而在对于疫情防控来说,隔离程度又是一个必然要解决的重大问题。我们有必要将其抽象成流动配置问题来进行统一处理。

\section{预期创新点}
全他妈是创新点。