\chapter{研究现状与问题归纳}

流行病学在数学上有着悠久的历史。现在的主流框架将问题归纳为两个层次:单点的疾病发展,以及疾病在网络框架下的传播。

\section{流行病的传统数学模型}

传染病的基本数学模型就是SEIR模型该模型假定人群分为4种,分别是:易感者(SUSCEPTIBLES), 潜在的可感染人群(EXPOSED):潜伏者, 已经被感染但是没有表现出来的人群;感染者(INFECTIVES), 表现出感染症状的人; 抵抗者(RESISTANCES), 感染者痊愈后获得抗性的人。亦有称R为RECOVERER的,也就是恢复者, 但是实际上如果是致死性疾病, 死者也是算进这一项里的, 毕竟死者妥善处理以后无法被感染也无法感染别人, 和恢复者是一样的。通过对这几种人群数量的动态演化观测,我们可以确定疾病传播的不同阶段,进而制定防疫策略。

\section{社会接触模型}

