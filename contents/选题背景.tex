\chapter{选题背景}

2020年一月,新型冠状病毒2019-nCoV流感在武汉开始肆虐,并迅速传遍全国。该事件映射出大城市在面对突发灾害时,应对能力之差,以及公共应对措施的匮乏。城市政策应该是规模、密度、形态三位一体的\cite{xiu_2003}。如何更好地理解城市在面对突发情况时合理的宏观应对措施,应是每个有地理学思维人的共同问题。

数学模型可以作为真实系统的一个很好的模仿。一个好的数学模型可以解释很多对真实世界的观测结果,给出洞见,并提升我们对系统本质的理解,对未来的决策也有指导意义。对于疾病传播来说,网络科学是一个比较合理的建模方式。我们也可以找到比较成熟的方式来对该类问题进行处理\cite{kiss2017mathematics}。