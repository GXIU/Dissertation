\chapter{选题背景}

2020年一月,新型冠状病毒2019-nCoV流感在武汉开始肆虐,并迅速传遍全国。该事件映射出大城市在面对突发灾害时,应对能力之差,以及公共应对措施的匮乏。城市政策应该是规模、密度、形态三位一体的\cite{xiu_2003}。如何更好地理解城市在面对突发情况时合理的宏观应对措施,应是每个有地理学思维人的共同问题。历史上,天花、黑死病、痢疾、霍乱等瘟疫都留下了惊人的死亡数字。而近期的SARS、禽流感、H1N1等疾病的流行也引起了世人的恐慌。 公元前 1100多年前,印度或埃及出现急性传染病天花。公元前3~前2世纪,印度和中国流行天花。公元165~180年,罗马帝国天花大流行,1/4的人口死亡。6世纪,欧洲天花流行,造成10\%的人口死亡。17、18世纪,天花是欧洲最严重的传染病,死亡人数高达1.5亿。19世纪中叶,中国福建等地天花流行,病死率超过1/2。1900~1909年,俄国因天花死亡50万人。霍乱于1817年首次在印度流行,1823年传入俄国,1831年传入英国。19世纪初至20世纪末,大规模流行的世界性霍乱共发生8次。1817~1823年,霍乱第一次大规模流行,从“人类霍乱的故乡”印度恒河三角洲蔓延到欧洲,仅1818年前后便使英国6万余人丧生。1961年出现第七次霍乱大流行,始于印度尼西亚,波及五大洲140多个国家和地区,报告患者逾350万。1992年10月,第八次霍乱大流行,席卷印度和孟加拉国部分地区,短短2~3个月就报告病例10余万,死亡人数达几千人,随后波及许多国家和地区。疟疾每年在全球有五亿宗病例,导致超过100万人死亡,大部份在非洲发生。世界卫生组织指出疟疾平均每30秒杀死一个5岁以下的儿童;疟疾也是导致非洲经济一直陷于困境的主要原因之一。公元前430~前427年,雅典发生鼠疫,近1/2人口死亡,整个雅典几乎被摧毁。第一次世界性鼠疫大流行;始于公元6世纪,源自中东,流行中心为近东地中海沿岸,持续近60年,高峰期每天死亡万人,死亡总数近1亿人。第二次世界性鼠疫大流行;史称“黑死病”,1348~1351年在欧洲迅速蔓延,患者3~5天内即死,3年内丧生人数达6200万,欧洲人口减少近1/4,其中威尼斯减70\%,英国减58\%,法国减3/4。此次“黑死病”延续到17世纪才消弭。第三次世界性鼠疫大流行;1894年,香港地区爆发鼠疫,波及亚洲、欧洲、美洲、非洲和澳洲的60多个国家,死亡逾千万人。其中,印度最严重,20年内死亡102万多人。流行性感冒简称流感,是由流感病毒引起的急性呼吸道传染病,能引起心肌炎、肺炎、支气管炎等多种并发症,极易发生流行,甚至达到世界范围的大流行。1918-1919年,爆发了席卷全球的流感疫病,导致2,000-5,000万人死亡,是历史上最严重的流感疫症。自2003年来全世界已有14个国家357人感染了禽流感病毒,其中219人因感染了该病毒而死亡。目前的H5N1型病毒株仅能通过禽类传染给人体,必须防范它与人类的流行性感冒病毒株接触进行基因重组,突变出“人传人”的禽流感病毒。禽流感一旦在人际传播,数亿人生命将受到威胁。HIV是艾滋病的病原体,主要通过体液、血液传播。 艾滋病联合规划署和世界卫生组织在“2006艾滋病流行最新情况”报告中说,世界上每隔8秒钟就有一人感染HIV,全球每天有1.1万人感染HIV,与此同时,每天有8000名感染者丧命。SARS(Severe Acute Respiratory Syndrome,严重急性呼吸道综合症,俗称非典型肺炎)是21世纪第一个在23个国家和地区范围内传播的传染病。2002年11月16日中国广东佛山发现第一个非典型肺炎的病例。截至2003年7月11日,全球共8096名患者,死亡人数达775,死亡率约为9.56\%。目前已经找到治疗方法,中国和欧盟科学家联手,成功找到了15种能有效杀灭非典病毒的化合物。香港大学的研究表明,蝙蝠可能是SARS病毒野生宿主。

流行病传播也跟随着一些地理空间上的思考。我们知道,流行病传播的过程,是通过人与人交互以及人在各种尺度上的移动实现的\cite{belik2011natural}。另一方面,疾病的控制也可以理解为缩减疾患出现的区域,使之不影响人们的正常生产生活。在历史上,我们也有一些成功的先例,利用地理信息系统的方式来对抗流行病的传播。我们举1832年英国霍乱的例子:英国医生约翰·斯诺(John Snow,1813-1858)发现,伦敦霍乱的大量病例都是发生在缺乏卫生设施的穷人区,他利用伦敦死亡登记中心的死者住址数据,将霍乱疫情的起源定位到了布劳德大街上的一口公共水井。这个发现使得水井被废除,疫情也得以消失。斯诺医生绘制的流行病地图是历史上影响最深远的可视化作品之一。这也是地理信息系统的思想的早期胜利。这说明用地理学的视角去理解疾病传播是有意义而自然的。

而随着城市化进程的不断推进,便捷的城市交通也加剧了流行病的蔓延的速度和控制的难度。出于对效率的追求,人类社会的诸多网络特征是无标度的。而根据网络上流行病学的基本结论:无标度网络上的流行病传播是不存在阈值的。即疾病最终会传染被网络连接的所有人\cite{PhysRevLett.89.108701,PhysRevE.65.035108}。流行病作为城市突发问题的典型代表,可以提示我们在城市区位设计的时候,不能完全遵循自组织的连接方式。其也应该有动态变化能力,以适应突发状况的出现。

数学模型可以作为真实世界很好的一个模仿。一个好的数学模型可以解释很多对真实世界的观测结果,给出洞见,并提升我们对系统本质的理解,对未来的决策也有指导意义。对于疾病传播来说,网络科学是一个比较合理的建模方式。我们也可以找到比较成熟的方式来对该类问题进行处理\cite{kiss2017mathematics}。已有的方法对于流行病的传播建立了一些有效的抽象模型。通过疾病传播过程中若干重要因素之间的联系建立微分方程加以讨论,研究传染病流行的规律并找出控制疾病流行的方法显然是一件十分有意义的工作。进一步,我们还将结合可达性与人类移动性的分析得出更多疾病控制的手段,为城市安全发展和城市韧性的提升提供合理的参考。