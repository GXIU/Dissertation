\chapter{Coevolution of Urban Heat Islands and Poverty}

The urban heat island (UHI) effect, a phenomenon that more elevated surfacial air temperatures in cities than that of their non-urban surroundings, is greatly due to the constructions linked with urbanisation~\cite{Yunfei2020On,Chapman2017The,zhao2014strong,Peng2012Surface}. The energy balance of urban areas differs from that of permeable stratum. First, its high thermal capacity, and thermal conductivity, urban areas concentrate heat that is hard to vanish; Second, the patterns of urban activities featuring residents and careers leads to unbalanced heat radiation as a consequence of heterogeneous exercise. Anthropogenic heat release from zonal activities also contributes to the variations of the accumulation of heat. When combined with the background of global warming, UHI exacerbates health risks~\cite{milan2015reducing}. Both morbidity and mortality of urban living increases due to the existance of the UHI effect~\cite{tan2010the,stone2015}.

Apart from health risks, the economic costs and the inequality created by the UHI effect and climate changes is estimated to be 2.6 times more than those without UHI~\cite{estrada2017a}. Measures to reduce the UHI impact will also contribute to heat stress mitigation and substantial developmental goals, especially in the future with more extreme heat events due to the exacerbation of urban activities on urban climate. As always been reported that mortality risk is significantly associated with minimum temperatures~\cite{kalkstein1989weather}, the inequality of the ability of dealing with it diverges, and may threaten the poors' situation even more.

The UHI intensity varies across and also within cities. Inner-city variation of the UHI may lead to different impact of urban heat stress on different demographic groups~\cite{chakraborty2019disproportionately}. It is also reported that in developing countries and regions the carbon emissions are higher than those in developed regions, implicating a co-evolutionary scheme between poverty and heat in cities. To examine these interinfluences, we combine UHI intensity measures with census data to evaluate the time-variant relationship between the geographical distributions of UHI and income at the scale of 100 meters from a multi-(American)-city perspective.

We find that in almost all cases, poorer neighborhoods and hotter areas coexist with high percentage; We also find that the two studied factors: urban heat tendency and urban poverty, actually coevolute over time. These findings suggests that authorities should consider building less discreminated reduction strategies to mitigate the impacts of UHI on the vulnerable groups who may suffer more from heat due to the radiation effects of poverty and heat and those less equipped to adapt to climatal stressors. One possible explanation is a neighborhood's vegetation density. Combining with the lawful setting of green covers in the United States, one strategy urban policymakers can consider is stricter zoning law to settle more economically disadvantaged residents on green-covered zones.

\section{Correaltion between UHIs and Poverty}

The chances in cities have driven more and more people into them, but also leads to the matter of the poors settle on heat islands in urban areas.

\begin{figure}
    \centering
    \includegraphics[width = 0.999\linewidth]{Pics/UHI_combined}
    \caption{UHI and poverty distribution.}
\end{figure}

\section{Coevolution of UHI and Poverty}

Why not