% 这个是第二章,第一节姚欣师兄写了七页半
\chapter{研究现状与问题归纳}

流行病学在数学上有着悠久的历史。现在的主流框架将问题归纳为两个层次:单点的疾病发展,以及疾病在网络框架下的传播。

\section{流行病的传统数学模型}

\subsection{SEIR模型的基本结论}

传染病的基本数学模型就是SEIR模型该模型假定人群分为4种,分别是:易感者(SUSCEPTIBLES),潜在的可感染人群(EXPOSED):潜伏者,已经被感染但是没有表现出来的人群;感染者(INFECTIVES),表现出感染症状的人;抵抗者(RESISTANCES)也即从疾病中恢复,并产生抗体的人, 但是实际上如果是致死性疾病,死者也是算进这一项里的,毕竟死者妥善处理以后无法被感染也无法感染别人,和恢复者是一样的。通过对这几种人群数量的动态演化观测,我们可以确定疾病传播的不同阶段,进而制定防疫策略。以下我们分别说明模型包含不同成分时的基本情况。

最基本的模型是SI模型。该模型描述了人群中被感染过的人的数量的变化规律。根据已有结论,SI模型中患病人数随时间的变化规律为Logistic曲线。该模型的经典结论是:所有人都会最终被感染;且疾病爆发时间出现在一半人口被感染的时候。用微分方程可以表示为
\begin{equation}
    \left\{\begin{array}{l}
    {\frac{d I(t)}{d t}=\rho I S} \\
    {S(t)+I(t)=1}
    \end{array}\right.
\end{equation}
其解为
\begin{equation}
    \left\{\begin{array}{l}
    {I(t)=\frac{1}{1+\left(\frac{1}{i_{0}}-1\right) e^{-\rho t}}} \\
    {t_{m}=\frac{1}{\rho} \ln \left(\frac{1}{i_{0}}-1\right)}
    \end{array}\right.
\end{equation}
该方法是不符合常识的:流行病通常不会使得所有人生病。实际情况中,有些被流行病感染的人会被治愈。病人恢复健康之后,可能会产生对该疾病的免疫力,也有可能反复被该疾病感染。这两种情况分别对应着SIR模型和SIS模型。

SIS模型是在SI模型的基础上,增加新的假设:单位时间内病人可以治愈的速率为$\alpha$,那么$1/\alpha$就可以理解为感染期。传染病康复后没有免疫力,仍然是属于易感者类型的。于是我们可以定义传染指数:一个感染期内单个病人的有效感染人数,\[\sigma = \rho/\alpha.\]该模型服从的微分方程是SI模型增添了一项恢复的速率,即\begin{equation}
    \left\{\begin{array}{l}
    {\frac{d I(t)}{d t}=\rho I S-\alpha I} \\
    {S(t)+I(t)=1}
    \end{array}\right.
    \end{equation}
这个模型是我们分析流行病传播的基准模型。该模型的基本结论是:患病人口长期状态下会趋于稳定,且是有效感染人数的函数。如果有效感染人数$\sigma\le 1$,则疾病最终会在人群中消失;反之,则患者群体会在人群中留有$1-\frac{1}{\sigma}$的比例。

SIR模型同样历史悠久\cite{kermack1927contribution}。该模型假设:除染病特征外,人群中的个体间没有差异;各类型个体在人群中混合是均匀的;总人数N不变,人数足够大,只考虑传染的平均效应;易感者染病的机会与他接触染病者的机会成正比;疾病的传染率是常数$\rho$,治愈率是常数$\alpha$。可以用微分方程表达为
\begin{equation}
    \begin{aligned}
    &\left\{\begin{array}{l}
    {\frac{d S}{d t}=-\rho I S} \\
    {\frac{d I}{d t}=\rho I S-\alpha I} \\
    {\frac{d R}{d t}=\alpha I}
    \end{array}\right.\\
    &S(t)+I(t)+R(t)=1
    \end{aligned}
\end{equation}
该模型的直观理解是:随着康复者R的不断出现,人群中感染者/可被感染者的数量在不断减少,直至消失。特别的,如果被感染者人数多于$1/\sigma$,则疾病会出现一个高峰;反之患病人数则会单调下滑到$0$。

\subsection{网络上的流行病学}

近二十年来,流行病作为人群中传播过程的重要实现方式,也在网络研究中被广泛讨论。人们关注的重点在于网络上流行病传播的阈值、收敛速度等问题。在网络上考虑流行病传播比较适合拓扑性质比较强的问题,比如疾病在大空间尺度上通过航空网络进行传播;在城市见通过铁路网络进行传播等。

目前也有一些工作将注意力转移到了空间网络上。这样的处理,更适合强调地理临近性的空间关系。

\subsection{人类移动性模式}



\section{社会接触模型}

\section{问题归纳}

结合地理大数据和空间交互模式研究可以对流行病传播的规律与控制流行病传播产生促进作用。这些规律亦有助于我们探讨城市在面临突发状况时,什么样的应对方式才是比较合理的。现有工作在归纳疾病传播一般机理上框架趋于完备。

\section{疫情的时空尺度}

在整个世纪以来,流行病模式发生了很大的变化。对于儿童传染病(如麻疹),模式转化主要存在于规则的周期和不规则的、可能是混沌的周期之间,以及从小区域同步振荡到复杂空间上的不连贯爆发。麻疹是一种自然生态系统,在不同的时间和地点表现出不同的动态过渡,但是所有这些过渡都可以由单个非线性模型的分叉来预测。B. Grenfell模型\cite{earn2000a}可以将这两种转变解释为出生和疫苗接种率变化的后果。