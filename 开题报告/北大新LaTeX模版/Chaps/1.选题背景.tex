\chapter{选题背景}
\setcounter{page}{1}
\pagenumbering{arabic}


\section{选题背景} % 要写三页

世界各国在快速城市化过程中,集中涌现出很多有共性的模式与问题。其中包括空气污染、城市热岛现象、能源消耗问题、社会空间不平等以及很多与可持续发展有关的问题。为了向决策者提供可靠的理论和新范式来缓解这些城市问题,对城市结构与演变规律进行建模成为尤为重要的一环。基于简单规则而可解析的统计物理模型可以在不同的角度给出城市发展机制的阐释和预判。近年来,可供城市研究参考的数据越来越多,建模的置信程度也在逐渐提高,这使得此类简洁有力的物理模型能够提供更好的连接理论和经验结果的工具和概念,从而在城市研究发挥重要作用。因此,构建可定量研究、能解释尽量多的城市现象、并有预测性的城市科学是城市研究中极具潜力的发展方向。

% 什么是介观,介观视角下城市的现象

介观尺度的概念由Van Kampen于1981年所创,指的是介乎于微观和宏观之间的尺度。针对城市科学,介观尺度是指城市内部几公里范围内的空间尺度,这是城市研究和城市规划中最常用的空间尺度。处于介观尺度的地学现象,尽管包含很多更小的研究单元,但其系统尺度小于相干尺度,介观尺度单元的同一样本中的空间单元保持较强的相关性,但各个空间单元的差异极大,城市系统的全局平均性质不再有效的刻画系统中所有微观空间单元的性质。理论角度看,在介观尺度下,城市的人口和社会经济密度保持了足够的异质性(如用地功能分化等);而对于该尺度的研究问题来讲,个体行为特征的聚合又可以体现出足够的规律性。实际角度看,介观尺度对于政策制定、感知城市空间、场所性等非具体、弱化实际度量的方面又有着很好的反馈效果。这使得基于街区、人口普查单元等介观尺度空间单元的研究体现出重要的意义。

% 为什么要用介观思维研究城市

城市是由环境和其中的人组成的。个体行为的总和构成了城市的动态。基于个体行为模式的挖掘,科学家在人类移动模式、职住平衡分析等方面建树颇多。然而,从数据来讲,由于个体数据采集成本较高、隐私保护等问题,基于个体的研究受数据制约比较大;从研究效果来讲,基于个体行为模式的研究结论整体可复现性不高,研究结论受随机误差影响比较大,同时空间一致性不强,很难得到较为通用的、有意义的模式。在此背景下,选取合适的研究尺度,从城市的内蕴框架下探究城市模式的涌现,就成了一个更好的选择。

% 广泛的适用性,普遍存在的问题

已有研究从介观尺度结构出发,已经挖掘出来很多有意义的模式。可以概括为如下三类:\begin{enumerate}
    \item 从空间分布(一阶量)来看,介观尺度模型可以反映空间上连续变化的事物的分布。在刻画空气污染\cite{mijling2012using}、城市热岛效应、城市形态演化\cite{raimbault2018calibration}、人口模式识别方面都是建模的核心。
    \item 从空间交互(二阶量)来看,介观尺度对应着聚合尺度的空间交互,在流行病传播、对政策制定等角度有着核心意义。% 传统数据比较匮乏
    \item 从介观尺度模式的形成来看,介观尺度观测很好地体现了城市微观-宏观之间的跨尺度特征,在交通拥堵、交互尺度提取等方面方兴未艾。
\end{enumerate} 

% 存在很多广泛的其他领域的方法可以处理其中的问题,但存在一些问题,使得已有的方法做出来的结果不能有普适性的意义。

社会物理(social physics)与地理大数据(big geodata)的结合有助于我们建立统一的框架来理解介观尺度规律。然而,这个框架中已有的方法和工具暂时不足以完全解决城市科学家的观测需要。其中几个主要方面为:\begin{enumerate}
    \item 数据空间与真实空间的非线性对应关系。
    \item 介观尺度结构空间交互的高耦合性。
    \item 统计涨落对城市介观尺度结构稳定性的影响。
\end{enumerate}

% 软与硬的对应:数据中硬的东西挖掘不出来了,需要理论高度和非线性映射来寻找城市内蕴的合适的结构。


% 因此,我们需要开发新的方法,