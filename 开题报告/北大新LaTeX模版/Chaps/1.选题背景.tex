\chapter{选题背景}
\setcounter{page}{1}
\pagenumbering{arabic}


% \section{选题背景} % 要写三页

% 什么是介观,介观视角下城市的现象
介观尺度的概念由Van Kampen于1981年引入,指的是介于微观和宏观之间的尺度\cite{van1992stochastic}。对于城市科学,介观尺度是指城市内部百米至十几公里范围内的空间尺度,这是城市研究和城市规划中最常用的空间尺度。微观的个体在小空间尺度交互,形成介观尺度的混合用地功能和特定的回转半径;介观空间结构作为城市整体的功能分化单元,通过其间的交互,完成城市整体的功能。处于介观尺度的地学现象,尽管包含很多更小的研究单元,但其回转尺度小于相干尺度,介观尺度单元的同一样本中的空间单元保持较强的相关性,但各个空间单元的差异较大:城市系统的平均性质无法有效的刻画系统中所有介观空间单元的性质。理论角度看,在介观尺度下,城市的人口和社会经济密度保持了足够的异质性(如用地功能分化等);而对于该尺度的城市现象来讲,个体行为特征的聚合又可以体现出足够的规律性。实际角度看,介观尺度对于政策制定、感知城市空间、场所性等非具体、弱化实际度量的方面又有着很好的反馈效果。这使得基于街区、人口普查单元等介观尺度空间单元及其之间交互的研究体现出重要的意义。

空间交互地理空间中不同位置的事物间存在不同强度的联系,并以物质、能量、信息等不同的形式进行移动和交换的过程\cite{liu1523}。空间交互的存在使得城市超越了个体的总和,让不同区位的市民有机地组合起来,完成了复杂的城市功能。在快速城市化过程中,全世界的大城市和城市群在结构和功能日趋复杂的同时,也出现很多有共性的模式,一些典型的例子包括单中心向多中心的转化、各种新的城市功能分区和空间组织结构的涌现;亦不乏影响城市可持续发展的问题,比如空气污染、城市热岛现象、能源消耗问题、社会空间不平等以及很多与可持续发展有关的问题。为了更新对当代城市的认识,并向决策者提供可靠的理论和新范式来缓解这些城市问题,我们有必要对城市结构与演变规律建立兼具解释性和预测性的模型。基于简单规则并且可解析的统计物理模型可以给出一些关于城市发展机制的阐释和预测。随着大数据时代的到来,理论模型的参数不断被校准,新现象、新机制也越来越多地在数据中被观测到。这使得此类简洁有力的物理模型能够提供更好的连接理论和经验结果的工具和概念,从而在城市研究发挥重要作用。这体现了构建可定量研究、同时具有可解释性和预测性的城市模型的正确性。具体来讲,城市是由环境和其中的人组成的。一定意义上来讲,个体行为的总和以及人地关系构成了城市的空间交互。经典的城市模型大多是基于个体行为模式的,近年来学界在人类移动模式、职住平衡分析等方面增添了很多认识。然而,从数据来讲,由于个体数据采集成本较高、隐私保护等问题,基于个体行为进行研究难以得到足够的数据支持;从研究效果来讲,基于个体行为模式的研究结论整体可复现性不高,研究结论受随机误差影响比较大,同时由于不同城市结构之间空间异质性的存在,很难得到较为通用的、有意义的模式。在此背景下,选取适当粗粒化的介观尺度进行研究,在城市的内蕴框架下探究城市模式的涌现,就成了一个更好的选择。本文将研究的交互问题分为两个尺度层次的含义:个体交互,即个体之间的见面、联系、相互影响;以及空间交互:定义在地理单元之间或者人流和物质交互。

% 广泛的适用性,普遍存在的问题

已有研究从介观尺度结构出发,已经挖掘出很多有意义的研究切入点。从蕴含要素的复杂性方面,可以概括为如下三类:\begin{enumerate}
    \item 从空间分布(一阶量)来看,介观尺度模型可以反映空间上连续变化的事物的分布,并在一定程度上克服噪声异质性的问题。通过预聚类/分类方法提出介观尺度空间划分的方式有助于我们得到更多规律性、模式性的结论,也能更好地摒弃预先栅格化城市空间所导致的研究谬误。因此,引入介观尺度概念描述分布问题在刻画空气污染\cite{mijling2012using}、城市热岛效应\cite{pi2019multi}、城市形态演化\cite{raimbault2018calibration}、人口模式识别\cite{dong2020understanding}方面都是建模的核心。
    \item 从空间交互(二阶量)来看,介观尺度对应着聚合尺度的空间交互。而在传统的研究中往往发现,行政区、街道等普通空间单元为基础的观测中,标度律、重力模型、辐射模型等规律往往难以得到很好的拟合效果\cite{jiang2011zipf, li2021gravity, mazzoli2019field}。这往往是由于行政区等空间划分方式失去时效性导致的\cite{jiang2011zipf}。新方法视角下介观尺度研究单元的建立有助于将交互的组内同质性和组间异质性的观察重新变为良定义的问题。在实际角度,这些方法在流行病传播、对政策制定等角度有着核心意义。% 传统数据比较匮乏
    \item 从介观尺度模式的形成来看,介观尺度观测很好地体现了城市微观-宏观之间的跨尺度特征。个体尺度的移动性规律规律是较为随机的\cite{gonzalez2008understanding},而群体尺度的交互规律则往往有着很强的规律性\cite{song2010modelling, rhee2011levy}。这种转变始于研究尺度的逐渐变大,而介观尺度如何吸收了个体的随机涨落,使人类行为体现出对称性,仍需要合适的个体交互模型来复现其规律。个体交互涌现介观一致性(consistency)问题\cite{riccardo2012towards}的探索 在城市分区功能分化、交通拥堵、交互尺度提取等方面方兴未艾。
\end{enumerate} 这些模式的提出隐含了介观尺度观测城市问题的普适性:既是个体交互的总和,又是构成城市整体性质的元素。

% 存在很多广泛的其他领域的方法可以处理其中的问题,但存在一些问题,使得已有的方法做出来的结果不能有普适性的意义。

社会物理(Social physics)与地理大数据(big geodata)的结合有助于我们建立统一的框架来理解介观尺度规律。然而,这个框架中已有的方法和工具暂时不足以完全解决城市科学家的观测需要。其中几个主要方面为:\begin{enumerate}
    \item 数据空间与真实空间的非线性对应关系。
    \item 介观尺度结构空间交互的高耦合性。
    \item 统计涨落对城市介观尺度结构稳定性的影响。
\end{enumerate} 首先,随着移动互联网时代的到来,地理学者可以在社会感知的框架下处理的数据不计其数。如果将这些数据汇总到一起,则凸显出数据中的结构,及其在真实空间中的对应,是很困难的。以英国的人口普查数据为例:每十年,政府会对每个目标区域统计约1450维变量。如何在研究特定问题时确定足够好的变量,以及他们的权重,是一个亟需解决的问题;其次,移动大数据观测了多尺度人类行为,但其中规律性的部分的提取、以及对随机性的预测仍然是衡量城市软性质的主要桎梏\cite{song2010limits,barter2019manifold,dalziel2013human}。最后,在拥有对城市生活和人类社会如此多的观测数据的前提下,很多城市模式的生成模型需要更多数据的检验\cite{makse1995modelling,wilson2003development,rozenfeld2008laws}。

% 软与硬的对应:数据中硬的东西挖掘不出来了,需要理论高度和非线性映射来寻找城市内蕴的合适的结构。

总之,大数据为建模城市提供了更多素材。但同时,提出更精确更合理的模型,以及检验以往猜测的准确性是进一步研究的重中之重。理解城市介观交互模式中涌现的软性质,并对其机理性提出解释,则是将城市研究科学化的核心观念。


% 因此,我们需要开发新的方法,