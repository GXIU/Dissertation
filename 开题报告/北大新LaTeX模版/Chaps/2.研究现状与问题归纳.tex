\chapter{研究现状与问题归纳}

\section{研究现状}

\subsection{介观尺度的城市内部交互问题}

% 个体与集体的移动性的区别
城市中的空间交互问题在不同的聚合尺度上通常有着不同的研究结论。以人类移动性研究为例:个体层次的移动性常用列维飞行来近似\cite{brockmann2006scaling,gonzalez2008understanding,mantegna1994stochastic,metzler2007some},其中体现了高度的随机性;而在群体层次,人类移动性则在出行目的、社群性质、频率分布等方面体现了较高的规律性\cite{peng2012collective, jiang2017collective, dong2020spectral, wachowicz2016finding}。这些群体的移动性汇集成了介观尺度空间单元的空间交互。随着交通、手机信令、社交媒体、细粒度普查等元数据采集精度的提高,从随机性的个体行为中提取群体行为并进行研究也有了足够的可行性。

城市化的进程伴随着以超线性速度增长的个体交互频率。在更高的交互频率下,城市生活提供了更多机会,例如更好的就业、更广泛的文化活动以及更多样化的人口。然而,这种交互增益也有负面影响:如犯罪问题更大,拥挤,低空气质量和较高的生活费用。许多研究表明,健康的社会结构在强调城市生活的积极方面起着关键作用。研究还表明,社会凝聚力可能受到设计决策的影响,如街道布局和交通限制\cite{appleyard1976liveable, appleyard1980livable}。我们可以区分那些被精确定义和中央管制的\textbf{硬属性},例如物质基础设施、交通规划和分区;和那些由居民的社会交往,例如社会凝聚力、城市精神和氛围构成的\textbf{软属性}。后者对于城市的居民幸福感、应对灾害能力、可持续发展、城市吸引力等方面都发挥着重要作用。因此,在发展基础设施建设的同时,考虑优化空间结构,提升交互性质,实现城市软属性的提升是一个有价值的目标。

% \subsubsection{城市标度问题}
\subsubsection{海量城市数据中找到关键变量}

通过数学模型,很多学者阐释了城市的软属性如何有序组织排列城市的硬属性,这些组织性质可以帮助决策者理解城市的韧性和健壮程度\cite{batty1971modelling, louf2013modeling}。经济学家使用大量的多变量模型来预测扰动对城市系统的影响,比如土地利用与交通的相互作用、产业结构调整对城市单-多中心模式转换的影响等\cite{fujita1982multiple, acheampong2015land}。这些预测可能非常准确,但依赖于对大量参数的拟合。而城市经济学家和经济物理学家更倾向于使用简单的规则和变量,基于城市居民的个体选择的群体效应来复现城市现象。这些模型体现了比较强的解释能力,但缺点是无法完全重现城市的多样性,以及难以证伪。因此简单规则导出的模型预测通常不适用于为具体的干预措施或政策提供建议。

另一个思路是从物理学借鉴方法论,考虑一小组经过仔细选择的变量建立动态模型,然后根据对少数变量的分析制定定性或定量预测。这种方法的优点在于结果不是从黑箱中产生的,而是通过理解过程的动态而获得的。对软城市特性的出现进行建模的一个核心挑战是识别一组能够捕捉这些特性的变量。通过将城市的文化和经济方面进行量化,对城市建模面临着无数的可能性:可以选取很多变量,如家庭收入、家庭规模、就业状况等,这些变量之间都有很强的相互依赖性。因此,挑战在于,在不产生不必要的冗余的情况下,选择一个理想的小变量来表示一个区域的状态。

随着近年来机器学习技术的普及与发展,以及社会感知等认知框架的建立。地理学界在地理数据挖掘领域有了广泛的应用。'城市数据挖掘'描述了一种方法论,以揭示一组地理空间数据内部的模式和规律的逻辑或数学和部分复杂的描述\cite{behnisch2009urban, bendimerad2016unsupervised}。基本的线性回归之外,诸如涌现自组织映射(Emergent Self Organizing Maps)、局部线性插值(Local Linear Embedding)等方法的引入逐渐使人意识到,在城市数据的参考系下,城市并不存在一个全局适定的度量。多维时空数据不均等地反映在不同区位的优势与劣势。进而使得以下成了很重要的问题:如何选择合适的统计指标来反映城市某个侧面的性质?

第一个尝试是手动选择关键指标。\cite{bosetti2020heterogeneity}中对流行因素和人口统计学指标进行了回归分析,证明了免疫力、人口分布、和人口流动的异质性的影响下,麻疹爆发的概率会显著被低估。但是,此类方法容易失去更深层次的因果关系。因此,我们需要一个客观的方式选择关键变量。这样,添加到模型中的每个变量都会最大限度地增加我们描述城市差异的能力。


第二种尝试是利用机器学习方法进行降维。其中代表性的方法为主成分分析。\cite{deng2008pca}利用多元卫星数据,结合了PCA方法进行了用地功能分类研究;\cite{nagendra2003principal}用PCA方法分析了高耦合性的交通、排放和气象数据,结果显示,24小时平均的四个主成分对交通和排放变量的载荷最高,且它们之间有很强的相关性。1小时和8小时数据的PC载荷表明它们之间的变化最小;\cite{palmason2005classification}分析了城市形态特征,得到了比较好的抽象表述。此类方法的优点是可以简单而直观地找到关键的数据法向量。但其有一个不容忽视的缺点:其无法很好地反馈城市中非全局、非线性的地理耦合关系。

近期,以扩散映射(diffusion map)\cite{barter2019manifold}为代表的流形学习方法在非线性地理大数据融合方面给出了一些答案。在多元/多源地理大数据的融合过程中,不同的数据来源会反映城市现象的一个方面,这使得在数据空间的城市形态可能是非欧几何:某些数据集或统计指标非常突出地显示城市的某个小区域的优点,而在其观测下大量的其他区域则非常均质。这使得定义一个局部度量显得尤为重要。在已有工作\cite{barter2019manifold}中,通过考察英国人口普查数据的1450个统计量的结果,作者通过扩散映射方法找到了最重要的两个城市特征对应的人群:大学生/大学雇员、以及保障性住房住户。该方法有很好的物理直观,也能找到数据中的非全局特征。根据我们与原作者的一些讨论,扩展该方法可以容易地得到更多城市特征,亦可以定量地解释人文地理中难以量化描述的城市现象,比如空间对偶\cite{lomi2000density}、城市多中心性等\cite{gordon1986distribution, mcmillen1997nonparametric}。


\subsubsection{空间交互视角下城市系统韧性与稳定性}

城市复原力是指城市系统及其所有组成部分在时间和空间尺度上的社会生态和社会技术网络在面对干扰时保持或迅速恢复到预期功能的能力,适应变化的能力,以及迅速转变限制当前或未来适应能力的系统的能力\cite{meerow2016defining}。城市地区除了作为资源消耗节点和创新场所之外,在理论和实践上都已成为人类系统弹性的实验室\cite{meerow2016defining}。Resilience 的词根来源于拉丁语单词 resilipo,意思是“反弹”。作为一个城市科学的术语,韧性也具有积极的社会内涵,代表着城市面临流行病、产业升级、区位重心转移等问题时,回到正轨的能力\cite{mcevoy2013resilience, o2013deconstructing}。高速增长的城市规模使得城市核心结构的人口和经济规模的密度到达了一个非常高的水平。如何在这种非均衡的系统中找到结构稳定性就成为了此背景下最重要的议题。一些城市系统的研究显示,城市现有交互模式被证明与贫民窟的形成有着很高的关系\cite{brelsford2018toward}。亦有研究表明,长期发展下,旧城市将不可挽回地被新城市取代\cite{fujita1997structural, cottineau2017diverse}。这些现状使得探讨未来城市设计时,有必要加入对“韧性”的讨论。

以流行病传播为代表的传播过程在城市交互系统中往往体现出复杂的性质。流行病传播的过程,是通过人与人交互以及人在各种尺度上的移动实现的\cite{belik2011natural}。另一方面,疾病的控制也可以理解为缩减疾患出现的区域,使之不影响人们的正常生产生活。在历史上,我们也有一些成功的先例,利用地理信息系统的方式来对抗流行病的传播。以1832年伦敦霍乱为例:当时大量病例都是发生在缺乏卫生设施的穷人区,他利用伦敦死亡登记中心的死者住址数据,将霍乱疫情的起源定位到了布劳德大街上的一口公共水井。这个发现使得水井被废除,疫情也得以消失。斯诺医生绘制的流行病地图是历史上影响最深远的可视化作品之一。而如今,贫民地区依然是疫情爆发的重灾区\cite{sahasranaman2021spread}。城市生态学者的一些研究中发现,除了卫生条件差之外,贫民地区在城市拓扑上也有着独特性质,通常是城市中流动性最差的区域\cite{brelsford2018toward}。这种局部强关联性更加催生了流行病的集中爆发。

在整个世纪以来,流行病模式发生了很大的变化。对于儿童传染病(如麻疹),模式转化主要存在于规则的周期和不规则的、可能是混沌的周期之间,以及从小区域同步振荡到复杂空间上的不连贯爆发。麻疹是一种自然生态系统,在不同的时间和地点表现出不同的动态过渡,但是所有这些过渡都可以由单个非线性模型的分叉来预测。B. Grenfell模型\cite{earn2000a}可以将这两种转变解释为出生和疫苗接种率变化的后果。而随着城市化进程的不断推进,便捷的城市交通也加剧了流行病的蔓延的速度和控制的难度。出于对效率的追求,人类社会的诸多网络特征是无标度的。而根据网络上流行病学的基本结论:无标度网络上的流行病传播是不存在阈值的。即疾病最终会传染被网络连接的所有人\cite{eguiluz2002epidemic, plucinski2013clusters}。流行病作为城市突发问题的典型代表,可以提示我们在城市区位设计的时候,不能完全遵循自组织的连接方式。其也应该有动态变化能力,以适应突发状况的出现。

% 城市韧性是未来城市可持续发展的新路径。然而,对城市弹性的认识和量化仍处于概念和探索阶段。在本研究中,我们提出了一个 "尺度-密度-形态 "的弹性框架以及指标模型,基于景观生态学和演化弹性的理论来研究城市弹性的演变。我们发现,空间发展是影响规模弹性的主要因素;人口分布与密度弹性显著相关;城市增长边界和生态基础设施是优化形态弹性的因素;"尺度-密度-形态 "的优良平衡促进了城市区域弹性的发展。我们给出了大城市发展弹性的建议,包括-防止城市无序扩张,控制建设用地规模;降低人口和建筑密度,促进低碳绿色生产和生活方式;加强生态网络建设;控制城市增长边界等。本研究希望能提供一个科学的空间指南,以实施弹性城市规划,并可作为城市弹性定量研究的案例。

城市内部组分之间的流量构成是异常复杂的,随着不同的观测尺度有着非常大的异质性\cite{masucci2013gravity}。

%%%%%%%%%%%%%%%%%%%%%%%

由于城市的形成是空间资源向中心集聚的过程,城市之间的距离经常不是空间优化的结果。这带来的后果就是城市群自然而然的有着拓扑网络的特征。社会网络的层次结构有着本质的空间特征。生态学中最初的层级结构就是用空间实例来描述的:岛屿上物种的类型是随着岛屿与陆地的距离而递减,因此岛屿上物种的组成是层层镶嵌的。系统稳定性的讨论始于\cite{may1972will}。该文章基于 Gardner 和 Ashby 实现的计算机模拟动力系统的结论:大型复杂系统只能在一定临界程度的连通性上面是稳定的。而城市系统的高联通度使其在面对外在冲击时有着天然的桎梏。虽然基于交互矩阵的生态系统结构分析已经有了比较悠久的研究历史,空间因素在生态中的真正意义还少有人解释。\cite{lin2019spatial}给了一个空间版本的社会困境的解答。其中隐含了两层深刻的意义:生态的时间性,系统在演化的过程中存在着周期性。这意味着区域发展过程存在着势能积累。另一方面,空间性给系统提供了缓冲空间,使得博弈的不利方的损失可以以类似“势能”的方式积累。这与系统的弹性\cite{gao2016universal}概念是不谋而合的。这个文章的结论可以解读为:自然系统的弹性来源于系统的空间性。

城市间交互的建立是由流达成的。城市间动态平衡是我们理想的状态。但是城市发展过程中出现的很多问题也是我们不能忽视的。经济周期的出现、城市收缩、城市发展不均衡等现象的出现使我们反思,城市发展中是否有着天然的陷阱,或者是难以避免的问题。我们在追求城市发展的时候,究竟在追求什么。公共地悲剧(tragedy of the commons,TOC)是Hardin\cite{hardin1968tragedy}探索并定义的一种社会困境,它发生在两个人或群体选择不同的策略来利用有限公共资源的时候。个人出于自身利益耗尽公共资源,综合收益将劣于个体间合作、有节制地利用资源得到的利益。这种问题的根源在于人类价值观的贪念使得增加公共资源的时候,增加的生产意愿不一定会提高。所谓“三个和尚没水喝”。城市发展也有着一些对应的场景。比如说城中村、贫民窟等现象,就是城市无法使得所有人都努力工作,却提供了过得下去的生活条件,使得贫民仍愿意居住在城市之中而导致的。有一些新的研究确定了一些条件。在这些条件下,城市虽然仍不会百分百随着自身的发展而一帆风顺、长治久安,但却能走出困境。


\subsubsection{个体交互作用下介观尺度模式的涌现}

% 回音室效应。
学界通常认为,城市空间结构是自组织的:即城市作为一个混沌系统,会自发形成平衡的耗散结构。在社会理论中,尼古拉斯·卢曼引入了“自我指涉” Self-referentiality的概念作为自组织理论的社会学应用\cite{luhmann1984soziale}。对于卢曼而言,社会系统的“交流”是自我复制的,即交流产生进一步的交流。因此,只要存在进化的交流,社会系统就可以自我复制。对于卢曼而言,人类就是系统环境中的传感器。这与我们熟知的社会感知框架也不谋而合。卢曼通过功能分析和系统理论发展出一套关于社会及其子系统的进化论。自组织现象意味着特定特征尺度的现象涌现。\cite{friesen2018similar, pelz2019similar}验证了全球主要大城市的贫民窟规模都是类似的,并会形成图灵不稳定性的图案;\cite{louf2013modeling, courtat2011mathematics}等工作通过交通与机会模式的平衡解释了城市多中心现象的出现;\cite{rosvall2005networks}根据人在路网中导航所需要的信息量证明,城市网络并不是为了沟通而优化的,而局部地理环境比网络拓扑有着更重要的意义。这些结论告诉我们,寻找驱动介观地理模式形成的原因有助于我们对城市组分交互的“属性”信息得到更好的先验知识。

“回音室效应”\cite{wang2020public, liu2020modeling}作为信息传播领域的一个研究热点,其中的研究方法和对人群的理解亦可以应用到理解介观模式形成问题中。由于人的从众心理和地理相关性的存在,地理空间单元容易产生极化和同质的意见集群,通常被称为回音室。回音室效应可能阻止人接触与其现有信仰相反的信息或观点,并进一步激化具有极端信仰的个人。在高度信息化的社会中,回音室效应在各种时空尺度加速出现\cite{wang2020public}。已有研究中,社会影响模型\cite{noah2006structural, friedkin2011social, parsegov2016novel},阿克塞尔罗德的文化传播模型\cite{axelrod1997dissemination},有界信心模型\cite{deffuant2000mixing}等理论工作确立了解释微观尺度偏差积累到宏观尺度的基础。这些结论与人类移动性模型的探索返回、偏好重访等机制又有着异曲同工之处\cite{song2010modelling, gonzalez2008understanding, brockmann2006scaling}。在这些人类移动的行为模式涌现的过程中,如何防止城市动态固化、局部同质化将是规划者保持城市活力的重要考虑方面。

为此,经济地理学家基于同质化人口给出了一些模型来定量解释城市模式固化的条件。% schelling

\section{问题归纳}

介观尺度聚合地理大数据和理论模型方法可以挖掘个体行为模式的确定性、城市的空间结构、以及城市作为一个系统的性质,有助于理解城市面对外来冲击、政策变化等因素的反应。不过,自适应空间单元的提取、城市交互系统稳定性指标、交互模式中的涌现现象等问题还欠缺足够好的处理方法。

\subsection{自适应空间单元的提取}

要深入地了解城市的社会结构,虽然现有丰富的数据源提供了很大的便利,但是这些数据本身的复杂性也是一个挑战。以英国的人口普查数据为例:人口普查主要统计数据和快速统计报告为每个人口普查产出地区提供1450种不同的统计特征。如何找到其中的主要变量,并摒弃选择的偏见,得到一个相对客观的结果,是一个不易解决的问题。