\chapter{研究现状与问题归纳}

\section{研究现状}

\subsection{介观尺度的城市内部交互问题}

% 个体与集体的移动性的区别
城市中的空间交互问题在不同的聚合尺度上通常有着不同的研究结论。以人类移动性研究为例:个体层次的移动性常用列维飞行来近似\cite{brockmann2006scaling,gonzalez2008understanding,mantegna1994stochastic,metzler2007some},其中体现了高度的随机性;而在群体层次,人类移动性则在出行目的、社群性质、频率分布等方面体现了很高的规律性\cite{peng2012collective, jiang2017collective, dong2020spectral, wachowicz2016finding}。随着交通、手机信令、社交媒体、细粒度普查等元数据采集精度的提高,从随机性的个体行为中提取群体行为并进行研究也有了足够的可行性。

在更高的交互频率下,城市生活提供了更多机会,例如更好的就业机会、更广泛的文化活动以及更多样化的人口。然而,这种交互增益也有一些缺点,如犯罪问题更大,拥挤,低空气质量和较高的生活费用。许多研究表明,健康的社会结构在强调城市生活的积极方面起着关键作用。研究还表明,社会凝聚力可能受到设计决策的影响,如街道布局和交通限制\cite{appleyard1976liveable, appleyard1980livable}。我们可以区分那些被精确定义和中央管制的\textbf{硬属性},例如物质基础设施、交通规划和分区;和那些由居民的社会交往,例如社会凝聚力、城市精神和氛围构成的\textbf{软属性}。后者对于城市的居民幸福感、应对灾害能力、可持续发展、城市吸引力等方面都发挥着重要作用。因此,在发展基础设施建设的同时,考虑优化空间结构,提升交互性质,实现城市软属性的提升是一个有价值的目标。

通过数学模型,很多学者阐释了城市的软属性如何有序组织排列城市的硬属性,这些组织性质可以帮助决策者理解城市的韧性和健康程度\cite{batty1971modelling, louf2013modeling}。经济学家使用大量的多变量模型来预测扰动对城市系统的影响,比如土地利用与交通的相互作用、产业结构调整对城市单-多中心模式转换的影响等\cite{fujita1982multiple, acheampong2015land}。这些预测可能非常准确,但依赖于对大量参数的拟合。而城市经济学家和经济物理学家更倾向于使用简单的规则和变量,基于城市居民的个体选择的群体效应来复现城市现象。这些模型体现了比较强的解释能力,但缺点是无法完全重现城市的多样性,以及难以证伪。因此简单规则导出的模型预测通常不适用于为具体的干预措施或政策提供建议。

另一个思路是从物理学借鉴方法论,考虑一小组经过仔细选择的变量建立动态模型,然后根据对少数变量的分析制定定性或定量预测。这种方法的优点在于结果不是从黑箱中产生的,而是通过理解过程的动态而获得的。对软城市特性的出现进行建模的一个核心挑战是识别一组能够捕捉这些特性的变量。通过将城市的文化和经济方面进行量化,对城市建模面临着无数的可能性:可以选取很多变量,如家庭收入、家庭规模、就业状况等,这些变量之间都有很强的相互依赖性。因此,挑战在于,在不产生不必要的冗余的情况下,选择一个理想的小变量来表示一个区域的状态。

% \subsubsection{城市标度问题}
\subsubitem{多元地理大数据的数据挖掘}



\subsubsection{空间交互视角下城市系统韧性与稳定性}



\subsubsection{个体交互作用下介观尺度模式的涌现}



研究表明,个体的流动性是由经常访问的一些空间邻近区域组成的。


\section{问题归纳}

介观尺度聚合地理大数据和理论模型方法可以挖掘个体行为模式的确定性、城市的空间结构、以及城市作为一个系统的性质,有助于理解城市面对外来冲击、政策变化等因素的反应。不过,自适应空间单元的提取、城市交互系统稳定性指标、交互模式中的涌现现象等问题还欠缺足够好的处理方法。

\subsection{自适应空间单元的提取}

要深入地了解城市的社会结构,虽然现有丰富的数据源提供了很大的便利,但是这些数据本身的复杂性也是一个挑战。以英国的人口普查数据为例:人口普查主要统计数据和快速统计报告为每个人口普查产出地区提供1450种不同的统计特征。如何找到其中的主要变量,并摒弃选择的偏见,得到一个相对客观的结果,是一个不易解决的问题。