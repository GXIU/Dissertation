\chapter{研究内容}

\section{介观尺度空间交互问题的三个切入点}
% 一页
针对城市中的各种交互问题,找到合适的研究尺度,并在该研究尺度所确定的空间划分上确定空间交互的描述是给出解答的重点。在两个极端的尺度(即个体时空行为所对应的微观尺度,与整个城市的系统性特征所对应的宏观尺度)之间,我们期待可以建立一套以把握介观尺度空间交互规律为纲要的方法论。

第一个切入点是针对问题的介观尺度空间区域的提取。我们在利用地理大数据进行研究时,采样精度往往与需要探究的问题不完全一致,需要我们聚合成针对问题的合适尺度,在本文中即介观尺度空间单元。我们利用\textbf{扩散映射}技术,在多元数据中找到合适的特征,以针对需要的问题得到合适的空间范围。扩散映射方法与传统方法(如PCA等)的不同之处在于不会将数据降维成特定一组变量的全局权重,而是找到异质性地理空间的局部相似性。一个典型阐释是:同一群人的不同出行目的使得两组微观空间单元分别有着相似的特征,但两组空间单元之间并不类似。而扩散映射可以较好地提取此类特征。另外,由于其结构并不人为选取变量,扩散映射所提取的索引不会引入源数据偏见之外的偏见,所以对于数据操纵的企图具有很高的弹性。

第二个切入点是介观尺度视角下城市系统稳定性。城市各组分沟通得紧密也伴随着城市面临外在冲击时,冲击在城市内部快速的传导。城市面对流行病等外来冲击时的抵抗能力通常被概括为城市韧性/稳定性。城市稳定性也对可持续发展的诸多方面(多样性,连通性,去中心化和自给自足)有解释作用。我们试图基于矩阵理论,利用城市中介观尺度空间单元的交互强度提出一种稳定性度量。在此理论下,可以通过求解矩阵的实部为负的特征值所占全部特征值的比例来衡量外来冲击在城市内部爆发的倾向性。

第三个切入点是介观尺度下的信息传播与共识达成问题。随着城市规模的增大,城市中各个区域的沟通也变得更加紧密。根据\cite{stark2008decelerating}, 更频繁的微观尺度交互会使得政策贯彻和城市面对外来冲击的弹性降低。为此,我们借鉴Hubbell模型\cite{hubbell2011unified},考虑不同社区中有若干个体,而每个个体针对一件时事(比如,疫情期间政府对于戴口罩的倡议)都有一个\textit{意见},在随机初始状态下意见统一的期望时间。我们发现,加入社区、集合种群等介观结构更容易达成意见统一。其他细致结构也对于我们理解介观结构对于形成城市意见共同体的影响。

根据我们的理论和实证结果,介观空间单元的交互视角下,城市的空间连续性、一致性、稳定性等特征的结果都与微观尺度交互的结果不同。对于地理分析来说,介观尺度空间单元的引入使得结果更加鲁棒,摒弃了微观尺度空间单元作为研究基础常面临的随机性问题和维数灾难;进而应用谱性质、动力系统的方法也可以绕过繁琐的计算而得到更清晰的洞见。
% 六七页
\section{具体研究内容}

\subsection{利用扩散映射提取城市隐藏特征}

从认知、规划等角度来讲,城市空间都不是一个典型的欧氏空间:同样的欧氏距离下,是否穿越河流、人流稀稠,都会产生不同的时间消耗和其他感知差异。这个事实说明城市空间的刻画是复杂的,全局的空间度量很难刻画城市空间的本质。为了更好的描述城市空间,我们需要更复杂的城市认知框架。数学上基于局部距离定义的\textit{流形}提供了一个可能的方案。

针对多元大数据对空间描述的空间异质性问题,我们引入并发展了扩散映射方法,通过将微观空间单元(即数据采集单元)在特征空间的相似性建模为对应复杂网络上的连边,进而用复杂网络的谱性质得到网络的若干长时间尺度特征描述。我们所使用的数据是英国和美国的的人口普查数据。

扩散映射是一种流形学习方法。其出发点在于:很多多元数据的数据元只能反映客体的一部分特征,只有将多个特征合起来才能得到一个比较完整的刻画。提取多个特征的“权重”并不是一件简单的事,因为很多时候数据的特征是非线性变化的。这种特征往往很难用一种全局度量来衡量。流形强调的是局部性质:流形 (Manifold) 是局部具有欧氏空间性质的空间,包括各种纬度的曲线曲面,例如球体、弯曲的平面等。流形的局部和欧氏空间是同构的。流形学习假设所处理的数据点分布在嵌入于外维欧氏空间的一个潜在的流形体上,或者说这些数据点可以构成这样一个潜在的流形体。流形是线性子空间的一种非线性推广。扩散映射近似流形的办法是在流形表面的建立复杂网络,并认为个体对这些区域的访问对应着复杂网络上的随机游走:类似人群总是访问网络距离比较近的一小片节点。相同人群访问位置通常对应着相同的频率等性质,所以网络的描述矩阵(Laplacian)不同特征值(频率)对应的特征向量可以区分不同区域某类型活跃人群的空间活动,进而得到人类行为对空间性质的解释。同一类人群活跃的区域中空间连续的部分即我们所定义的介观尺度空间单元。

具体实验流程方面,基本的过程如下:对于英美两国的普查数据,我们首先将所有的人口普查单元上的$k$个特征作为行向量,所有的普查单元作为列向量,构成一个矩阵$A$。我们的研究分为下述两个部分:\begin{enumerate}
    \item 找到普查规律的主要贡献变量;
    \item 优化普查指标选取,去除冗余变量。
\end{enumerate}
第一个部分的技术路线如下:\begin{enumerate}
    \item 将矩阵$A$进行列标准化;
    \item 将$A$的每一行作为一个向量,求矩阵两两行之间的欧氏距离,作为区域之间邻接矩阵的边权重;
    \item 以一个阈值$p$, 将权重过低的边去掉,得到截断Laplace矩阵$L$;
    \item 对矩阵$L$求特征值和特征向量,并以最小的正特征向量作为每个人口普查单元的权重,得到扩散映射的结果。
\end{enumerate} 通过该方法,我们找到了最小的两个特征值对应的特征向量指示的部分英国城市最重要的贡献指标:大学生和教员的密度,以及保障性住房的空间分布。

第二个部分则是反过来,将每个特征看作一个节点,将微观空间单元看作对该特征的描述,将有类似解释能力的特征聚合成“介观特征单元”。技术路线如下:\begin{enumerate}
    \item 将矩阵$A$求转置得到$A^T$进行列标准化;
    \item 将$A$的每一行作为一个向量,求矩阵两两行之间的欧氏距离,作为指标之间邻接矩阵的边权重;
    \item 以一个阈值$p$, 将权重过低的边去掉,得到截断Laplace矩阵$L$;
    \item 对矩阵$L$求特征值和特征向量,并以最小的正特征向量作为每个统计指标的权重,得到扩散映射的结果。
\end{enumerate} 该方法可以使我们优化人口普查的结果,避免冗余特征对实验结果的影响。

针对美国普查数据的初步结果如图\ref{fig:diffusionmap}所示。

\begin{figure}[h]
    \centering
    \includegraphics[width = 0.99\linewidth]{Figs/diffusionmap.png}
    \caption{扩散映射方法下,人口统计相关矩阵的前三小的正特征值对应的特征向量的空间分布。分别对应着纽约州的搞笑分布、贫困分布、和游客景点。}
    \label{fig:diffusionmap}
\end{figure}

而援引\cite{barter2019manifold}中英国布里斯托和爱丁堡两个城市的结果如图\ref{fig:manifoldcities},英国普查数据集进行扩散映射的结果更为明显,得到的介观空间模式更清晰:最小正特征值对应的特征向量揭示了大学生活动范围的若干介观空间单元。

\begin{figure}[h]
    \centering
    \includegraphics[width = 0.99\linewidth]{Figs/manifoldcities.png}
    \caption{英国布里斯托和爱丁堡的扩散映射结果,该城市对普查数据的空间差异性贡献最大的是大学生的空间分布。}
    \label{fig:manifoldcities}
\end{figure}

关于扩散映射的数学原理,可参考本报告的附录。


\subsection{城市韧性的动力学定义和特征}

本研究借鉴了生态学的“稳定性”概念\cite{may1972will, allesina2012stability},拟提出一种衡量城市系统对抗外来冲击的指标,以判别流行病、谣言等外来因素在连通度日益提高的城市系统中是否会扩散开来。该框架的建立可以衡量城市系统的各个子系统的保守性,进而识别出城市中易于受攻击的子区域。该工作可以很大程度上取代复杂的多主体建模,而得到更鲁棒性的结果。

生态学中稳定性衡量一个系统各组成部分的均衡种群数量如何由其相互作用影响。如果系统是稳定,则系统在经历扰动之后可以自发恢复到原有的种群密度;反之,则系统不能恢复原有状态。数学上,系统是稳定的,如果系统的全部特征值的实部都小于$0$,这对应着全局的负反馈\cite{coyte2015ecology}。城市系统中过多的涌现现象亦可以用其缺乏足够的稳定性来解释:城市流动网络经常无法幸免传染病和流言的爆发。这种性质通常是由于城市系统中过多的“节点”处于正反馈的网络路径上。外来冲击如果击中网络中的正反馈路径,则其会通过人类移动性网络快速传播。

我们试图研究:介观交互系统如何吸收外来的“冲击”?我们使用一个分析框架来量化介观交互系统的稳定性,衡量其是否能在受到冲击后自动让冲击退散(fade out),并给出了一个达到完全稳定的充分条件。最后,我们将城市内疫情爆发作为介观交互系统对外部冲击的不稳定诱导反应的案例进行研究,控制介观尺度空间单元与自身的相互作用对于稳定疫情传播具有突出作用。因此,介观尺度空间单元内部的保守性(表现为医院总病床数等“反”传播特性)保证了介观交互系统的弹性,对它的控制增强了我们应对外部冲击的能力。

该工作的数学基础如下:介观尺度上,城市内部的人类移动性可以理解为介观尺度空间单元$i = 1,2,\dots,n$之间的交互流,进而可以抽象成一个交互矩阵$W$, 其中矩阵$W$的元素$w_{i j}$代表空间单元$i$到空间单元$j$的流量。外来的冲击改变系统是由冲击系统的局部开始的,即各个介观尺度空间单元受影响的人口$Y_1>0$, 而$Y_n = 0, n = 2,3,\dots, n$. 以流行病传播为例:某个介观尺度空间单元出现了病例,病例进而随着交互流和局部扩散传播。疫情是否会扩散开,取决于患病子人口$(Y_1,\dots,Y_n)^T$在矩阵$W$作用下的稳定性:如果稳定性高,则疫情流行度会维持在$0$附近,即不会爆发;如果稳定性低,则疫情会以很高的概率爆发。鉴于不同的城市特征和流行病等冲击的不同,不同城市的交互流对外部冲击的反应非常异质。然而,从统计角度来看上看,城市系统的交互模式有几个共同的特征:首先,人们经常去人口密集或热门的地方,使得这类地方成为控制冲击的关键位置;其次,停留时间加权下,局部交互占总交互的比例对于不同的城市系统都保持在10-20\% 左右,但是,城市在面对疫情等冲击时,短距离的交互会因为补偿效应而占有更高的比例(如图\ref{fig:allee1}f)。这些人类流动的特征都会影响社会在面对外部冲击时的稳定性。基于以上的观察,我们建立了一个模型来解释外来冲击是如何随着人类流动而积累的。每个介观尺度空间单元受冲击的人口数随时间的变化可以由下述方程确定:\begin{equation}
    \begin{split}
        \frac{d Y_n}{d t} = \alpha X_n (w_{nn} \theta Y_n / N_n + (1-\delta) \Delta F_n^I) - \beta Y_n, \\
        \text{其中}\quad \Delta F_n^I = \sum_{m \ne n} (w_{mn} Y_m / N_m - w_{nm} Y_n / N_n)
    \end{split}\label{eq:allee_basic}
\end{equation}是介观尺度空间单元$n$受其他介观尺度空间单元冲击的人口数,$\theta$和$\delta$分别代表了外在冲击影响下人类移动性在局部和全局的扩散系数。传播速度和遗忘速度则分别是$\alpha$和$\beta$. 利用一个疫情传播的易感-患病-移除(SIR)场景下的符号系统,我们将受冲击子人口的交互矩阵近似为\begin{equation}
    A^t_{m n}=\begin{cases}
\alpha(1-\delta)\left[ w_{m n} I_{m} / I_{n}N_{m} - w_{n m} / N_{n}\right] & m \neq n, \\
\theta \alpha S_{n}-\beta & m=n.
\end{cases}
\end{equation}最后,我们定义城市交互系统的稳定性为外来冲击下,受冲击人口$Y$的交互矩阵$A$的特征值的实部的和是否为负

\begin{figure}
    \centering
    \includegraphics[width = 0.9\linewidth]{Figs/Figure1.jpg}
    \caption{\textbf{城市稳定性的框架图}。\textbf{a-b}, 介观交互系统(城市)面对外部冲击时的反应。冲击从一些元人口渗透到城市中,并通过经常性的流动网络和地方互动传播。\textbf{c}, 城市的抽象流动矩阵:在冲击期间,社区内部的访问更为频繁,对应$\theta > 1$. \textbf{d}, 系统的稳定性与其抵抗冲击转为爆发的能力之间的负向关系。 \textbf{e}, 华盛顿特区个体流动性的半径分布。最冷的颜色代表2020年的第一周,而最暖的颜色代表最新的。考虑到COVID-19的冲击,跨社区个体流动性的规模下降,恢复模式不变。\textbf{f},2020年1月至6月1日美国25个城市人口普查区间和人口普查区内流动意向的周比值。趋势代表了城市地区个别的运动半径,在2月中旬达到顶点,即左边的虚线,直到3月下旬才下降,即中间的虚线。这一结果支持了局部相互作用的作用在检疫期增加的说法。}
    \label{fig:allee1}
\end{figure} 该稳定性的优点是给定任意的政策条件,稳定性都可以在秒级时间得到结果,进而可以定量比较不同控制政策和疏导效果对某个城市的影响程度,也使不同城市之间交互流的不同有了一个统一的建模方式。该工作的框架如图\ref{fig:allee1}所示。

依照这个框架的建立,我们还可以得到一些理论上的指引,比如控制城市内传播现象的一个充分条件就是每个介观单元可以完全吸收(absorb)单元内部的扩散速度;已被影响人口的密度达到高点时,稳定性会相对较低等。我们根据实证数据\cite{kang2020multiscale}对不同时期美国主要城市的稳定性进行了一些验证,如图\ref{fig:allee2}所示。我们发现控制短程交互和长程交互对疫情传播有着反常的影响:短程交互所起到的作用更大;我们的方法也有助于寻找最优停摆区域;而疫情传播的稳定性周期大约在三周左右。我们也探讨了不同程度控制政策,以及居民反馈程度对疫情传播的影响,如图\ref{fig:allee3}所示。疫情流行程度高的时候(11月),稳定性的等势线较为水平,意味着控制长程交互基本已经无法对疫情传播起到控制作用,只有彻底限制居家才能起到控制的作用。

\begin{figure}[h]
    \centering
    \includegraphics[width = 0.6\linewidth]{Figs/Figure2.jpg}
    \caption{冲击下人类流动模式的特征会影响介观交互系统的稳定性。 \textbf{(a)} 一个城市(华盛顿特区,7月)在不同程度的限制(影响非本地流动)和回应(影响本地流动)下的快照稳定性(即以周记)。\textbf{(b)} 考虑 "较小 "的系统,屏蔽掉随机挑选的$x$个人口普查区(暗线为 10 次试验的平均值)和$x$个最关键的社区。 \textbf{(c)} 美国多个城市每周的稳定性变化的典型的时间序列的具有典型的两或三周的周期。}
    \label{fig:allee2}
\end{figure}


\begin{figure}[h]
    \centering
    \includegraphics[width = 0.9\linewidth]{Figs/Figure3.png}
    \caption{左栏分别是华盛顿特区大都市统计区5月、7月、9月、11月发病率的热图。中间一列包括社区矩阵在复数平面上的特征值的相应核密度。右列是华盛顿特区的稳定性相图,它是全局尺度交互限制$\delta$和市民反应$\theta$的函数。冷色表示给定感染者的空间分布时较低的稳定性。}
    \label{fig:allee3}
\end{figure}

\subsection{从众心理驱动下的介观模式形成问题}

% 这个工作很有意义,建立微观和介观的桥梁。从完整性上,介观和宏观之间的关系如何?如何从介观参数得到宏观特征?

% 但有个问题,就是整个工作没有涉及空间。就是没有落到空间上,基本上是一个社交网络的工作。反而你举的贫民窟例子挺好的。

% 另外,工作内容二和三和第一项之间没有关系,尽管从理论角度讲第一项是基础。

城市中的多种传播过程,比如流行病、谣言、方言、城市功能分化、社区形成等,都由同样的数学机制支持\cite{gao2019effects, ribeiro2020city}。这一节中,我们以流行病的防疫意识的传播为例,讨论意识在不同性质的介观单元内如何达成统一共识(consensus)。

我们分别探讨两种传播过程:流行病的传播,以及以戴口罩为例的防疫意识的传播。流行病传播最简单的情形是:把复杂的空间交互行为还原成个体之间逐对的交互。这种方式可以概括为质量作用近似(mass-action approximation)\cite{mollison1995epidemic}。这个假设考虑的是一个随机混合的人口,忽略了家庭结构、社会集会和不同个体的不同行为。因此,质量作用模型有比较大的局限性,因为它们只关注每个病例引起的平均感染数量,即基本传染数$R_0$ ,而忽略了潜在的微观尺度社区结构的异质性\cite{hebert2020beyond}。在依赖质量作用假设的情况下,设计有针对性的干预措施也十分困难。已有研究通过引入高阶接触模式(比如社群结构中,更长接触时间所导致的更强传染性的全连接子交互结构,对应网络中的全连接三角形、全连接四边形等)来解决这些问题\cite{iacopini2019simplicial}。

心理因素也在流行病等传播过程中起到了重要作用:在疫苗尚未完全普及的现在,保持社交距离、公共场合佩戴口罩等非药物干预(non-pharmaceutical interventions, NPIs)以及人们对其的态度对防疫效果有着决定性的地位。很多国家的早期宣传里,戴口罩是只有患者需要做的防疫措施。这使得社会形成戴口罩的共识变得异常困难\cite{lai2020effect, van2020face, Adolph2020PandemicPT, hellewell2020feasibility, Wolf2020AwarenessAA, Cheng2020TheRO, eikenberry2020to, erku2020fear, enberg2020covid}。这种误导或认识的时空差异有时甚至来自权威机构和研究者,他们对口罩供应缺失的强调所产生的对大众会产生心理影响\cite{biancovilli2020governments,landi2020should,sugaya2020real,weill2020social},并通过从众心理进一步传播。理论上,通过统计物理学和心理学对促进健康关注的集体适应行为进行了广泛的研究~\cite{castellano2009statistical, centola2007complex, centola2010spread, centola2011experimental, christakis2007spread}。最核心的问题之一是,一个失衡的系统如何快速得到有序的发展。流行病的许多全球特征,如基本繁殖数$R_0$,可能起源于社会群体的异质混合,对非药物干预的接受程度不同。比如戴口罩在被证明对COVID-19等呼吸道疾病有效,但对戴口罩的误导和歧视依然存在,这使得戴口罩的社会共识极难形成,反而加剧了社会关系的异质性,即形成两个比较离散的戴口罩-不戴口罩的组团。话语和顺应性使得社会对戴口罩难以达成共识,可能导致社会意识的分裂~\cite{holme2006nonequilibrium}。

我们提出了一个异质性网络传播模型:选择戴/不戴口罩行为的个体对于流行病的传播速率是不同的。我们假设个体选择戴口罩的变化是由从众心理驱动的,而流行病的传播则是通过研究易感者-感染者-易感者(SIS)模型进行的。前者通过个体交互网络的三角形结构,向少数意见者传播;后者则通过边直接进行传播。从众心理的假设由于其在类似健康行为的传播得到了实验验证~\cite{christakis2008collective, zhang2016support},所以是十分合理的。在模型中,个体通过联系网络中的边进行连接。每一随机时刻我们随机选中一条边,如果其上两个节点的健康状况为易感-感染(S-I),则流行病以一定概率传播给易感者,概率为$\beta$。这个概率与两个人是否戴口罩是相关的。另外,每个随机时刻我们选中一个网络中的三角形,如果其上三个人有一个人与另外两个人对于戴口罩的意愿不同,则其以一定概率被同化,概率为$\beta_\triangle$。我们在各类接触网络(合成网络和现实世界的网络)上实现了这一点,并比较了分析平均场方法近似和基于个体模拟的传播结果。

该工作导出了一个临界条件,即初始戴口罩人数比例为$p_1$时,网络动态会收敛到全体都戴口罩;初始比例为$p_2<p_1$时,网络会处于震荡共存状态;初始比例为$p_3<p_2$时,网络会收敛到全体不戴口罩。该工作对介观尺度模式形成问题的高阶交互进行了建模,得到了一些有意义的结论。我们观察到,在某些网络初始结构下,初始愿意戴口罩的比例小于$0.51$时,不论后续意见传播速度如何,社交网络都会收敛到全员不愿意戴口罩的状态。如图\ref{fig:masksketch}所示。

\begin{figure}
    \centering
    \includegraphics[width = 0.45\linewidth]{Figs/masksketch.png}
    \includegraphics[width = 0.45\linewidth]{Figs/maskresult.png}
    \caption{左图:意见与疾病共同演化模型。戴口罩的意愿通过三角形向少数群体传播;其上还存在一个易感-感染-易感(SIS)的流行病传播模型。右图是相同初始状态下(意见传播速率$\beta_\triangle $, 疾病传播速率$\beta$), 愿意戴口罩人的比例随着时间的变化的平均场近似。}
    \label{fig:masksketch}
\end{figure}

该工作的实验结果为介观空间模式的形成提供了一些参考:不同介观模式的形成很可能是“初值敏感”的,给出初始政策建议、规划意见时,我们要注意规避这些敏感的初值位置,避免不戴口罩的集群、不良空间结构(比如贫民窟)的出现。我们后续还会将真实交互数据融入进来,并将问题扩展到贫民窟模式的形成等问题上。将微观尺度数据和介观尺度模式的形成的联系建立较为通用的框架。



\section{预期创新点}

本文通过引入和构建较多的数学工具,对城市介观尺度下空间模式的提取、交互模式的性质、局部模式的涌现进行研究。本文的几个主要结果改进了地理大数据预处理的设计,提取了合适的介观尺度空间结构;提出了定量衡量城市政策效果的方法,证明了介观尺度模式涌现的必然性。预期创新点如下:\begin{enumerate}
    \item 改进扩散映射方法,引入负相似性的概念,建立更广义的城市组件关联关系,挖掘了更多的诸如对偶结构、收缩城市结构等城市内蕴结构;
    \item 建立了以流行病传播为例的介观尺度城市稳定性/韧性量化框架。该框架测度了城市面临疫情等外来冲击时的抵抗能力,以及各种时空尺度的政策在控制冲击时的效果。同时该框架亦可发现冲击扩散的关键节点;
    \item 通过考察一类较为通用的复杂网络动态及其高维组织结构,得到了佩戴口罩行为的初值敏感条件,并易于进一步扩展得到多因素共同影响的介观尺度模式形成问题的一般理论形式。亦用一个局部交互模型产生了足够好的“回音室效应”效果。为解释城市介观层次固化(fixation)的形成问题提供了理论基础。
\end{enumerate}